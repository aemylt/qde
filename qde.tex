% TeX'ing this file requires that you have AMS-LaTeX 2.0 installed
% as well as the rest of the prerequisites for REVTeX 4.1
%
% See the REVTeX 4 README file
% It also requires running BibTeX. The commands are as follows:
%
%  1)  latex qde.tex
%  2)  bibtex qde
%  3)  latex qde.tex
%  4)  latex qde.tex
%
\documentclass[%
 reprint,
%superscriptaddress,
%groupedaddress,
%unsortedaddress,
%runinaddress,
%frontmatterverbose, 
%preprint,
%showpacs,preprintnumbers,
%nofootinbib,
%nobibnotes,
%bibnotes,
 amsmath,amssymb,
 aps,
 pra,
%prb,
%rmp,
%prstab,
%prstper,
%floatfix,
]{revtex4-1}

\usepackage{graphicx}% Include figure files
\usepackage{dcolumn}% Align table columns on decimal point
\usepackage{bm}% bold math
\usepackage{hyperref}% add hypertext capabilities
%\usepackage[mathlines]{lineno}% Enable numbering of text and display math
%\linenumbers\relax % Commence numbering lines

%\usepackage[showframe,%Uncomment any one of the following lines to test 
%%scale=0.7, marginratio={1:1, 2:3}, ignoreall,% default settings
%%text={7in,10in},centering,
%%margin=1.5in,
%%total={6.5in,8.75in}, top=1.2in, left=0.9in, includefoot,
%%height=10in,a5paper,hmargin={3cm,0.8in},
%]{geometry}

\usepackage{amsthm}
\newtheorem{definition}{Definition}[section]

\begin{document}

\title{Quantum Digital Signatures}

\author{Dominic Moylett}
 \email{dominic.moylett@bristol.ac.uk}
\affiliation{%
 Quantum Engineering Centre for Doctoral Training\\
 University of Bristol
}%

\date{\today}% It is always \today, today,
             %  but any date may be explicitly specified

\begin{abstract}
While classical cryptography has relied on the assumed existence of one-way functions for proofs of security, quantum cryptography has given us information theoretically secure secret sharing. This has been achieved via quantum key distribution, utilising the uncertainty principle to limit what an eavesdropper can discover about a shared private key. Another application where we can have provably secure primitives is that of quantum digital signatures, which are used for authenticating that a message did come from a specific user. In this essay, we summarise the theoretical and experimental achievements in quantum digital signatures, and discuss their current limitations.
\end{abstract}

%\pacs{Valid PACS appear here}% PACS, the Physics and Astronomy
                             % Classification Scheme.
%\keywords{Suggested keywords}%Use showkeys class option if keyword
                              %display desired
\maketitle

%\tableofcontents

\section{Introduction}

\section{Preliminaries}

\begin{definition}
A function $f:\mathbb{N}\rightarrow\mathbb{R}$ is negligible iff $\forall~ c > 0 ~\exists~ n_0 \in \mathbb{N}$ such that $\forall~ n \geq n_0, |f(n)| < \frac{1}{n^c}$.
\end{definition}

The most common example of negligible functions are functions of the form $2^{-p(n)}$ for some positive polynomial $p$.

\section{Definition of Digital Signatures}

A digital signature scheme consists of the following stages:

\begin{description}
\item[Key Generation]Alice generates a signing key $sk$ and a verification key $vk$. The signing key is kept to herself, while the verification key is shared with other parties.
\item[Message Signing]Alice takes a message $m$ and her signing key $sk$ and produces a signature for that message $s_m$.
\item[Message Verification]Another party, Bob, takes a message $m$, Alice's verification key $vk$ and a signature $s_m$ and accepts if they think the message was sent and signed by Alice, otherwise they reject.
\end{description}

Another definition of verification used in, for example, \cite{quant-ph/0105032} has three different options for Bob. In this definition Bob can output:

\begin{description}
\item[1-ACC]He thinks the message-signature pair is valid and other parties will also find the message to be valid.
\item[0-ACC]He thinks the message-signature pair is valid and other parties will not find the message to be valid.
\item[REJ]He thinks the message-signature pair is invalid.
\end{description}

For security, we care about two properties with digital signatures:

\begin{description}
\item[Repudiation]The probability that Alice is able to convince Bob that a message-signature pair is valid and convince another party Charlie that the same pair is invalid is negligible in terms of the security parameter (often the length of the signature).
\item[Forgery]The probability that Bob can in polynomial time create a valid message-signature pair he has not previously seen sent by Alice is negligible in terms of the security parameter.
\end{description}

\section{Lamport's One-Time Digital Signature}

The original quantum digital signature was inspired from a classical family of digital signatures. This family of signatures were first published by Lamport\cite{lamp79}, and are reliant on the existence of one-way functions, defined below.

\begin{definition}
Let $\mathcal{X}, \mathcal{Y}$ be arbitrary sets. A function $f:\mathcal{X} \rightarrow \mathcal{Y}$ is one-way iff $f$ can be computed in polynomial time, but for any polynomial-time randomised adversary $f^{-1}:\mathcal{Y} \rightarrow \mathcal{X}$ and uniformly selected $x \in \mathcal{X}, \mathrm{Pr}[f(f^{-1}(x)) = f(x)]$ is negligible in terms of the security parameter (often the length of $x$ in bits).
\end{definition}

For a one-way function $f$, then Lamport's one-time digital signature scheme is defined as follows for Alice signing an $m$-bit message:

\begin{description}
\item[Key Generation]Alice generates $m$ pairs of uniformly selected random integers $\{k^i_0, k^i_1\}$. These will act as Alice's signing key, while the verification key will be the pairs $\{f(k^i_0), f(k^i_1)\}$ for $i \in \{0,...,m-1\}$.
\item[Message Signing]For each bit $m_i$ of Alice's message, she sends $k^i_{m_i}$ as her signature for that bit.
\item[Message Verification]Given Alice's verification key $\{v^i_0, v^i_1\}$ and her signature $k^i_{m_i}$, her message $m_i$ can be verified by accepting if $v^i_{m_i} = f(k^i_{m_i})$ and rejecting otherwise.
\end{description}

Because this signature scheme is classical and deterministic, it is trivial for any verifying parties to share their copies of the message and verification keys and thus impossible for Alice to repudiate the message. As for forgery, if it is possible for an adversary to forge a message given only one message from Alice then it is possible to invert $f$, thus $f$ is not a one-way function.

Note that this signature scheme can only be used for verifying a single $m$-bit message. This is because if an attacker has signatures for two different messages $M, M'$ then they can construct the valid signature for the new message $m_0m_1...m'_i...m_{m-1}$, where $m_i \neq m'_i$.

\section{Quantum Digital Signatures}

\subsection{Gottesman and Chuang}

The first quantum digital signature was devised in 2001 by Daniel Gottesman and Isaac Chuang \citep{quant-ph/0105032}. The protocol used the same principles as Lamport's one-time signature scheme, but utilised quantum one-way functions, which map classical $L$-bit strings $k$ to $n$-qubit quantum states $|f_k\rangle$. The full protocol is described below, where $f$ is the quantum one-way function, $c_1, c_2$ are security thresholds such that $0 \leq c_1 < c_2 < 1$, $M$ is the security parameter, and $T < L/n$ is the number of copies of each key:

\begin{description}
\item[Key Generation]To sign a single bit, Alice generates pairs of $L$-bit strings $\{k^i_0, k^i_1\}, i \in \{0,...,M-1\}$ uniformly at random to act as her private key. Her public key is the pairs of quantum states $\{|f_{k^i_0}\rangle, |f_{k^i_1}\rangle\}, i \in \{0,...,M-1\}$.
\item[Message Signing]To sign a bit $b$, Alice sends the tuple $(b, k^0_b, k^1_b,...,k^{M-1}_b)$.
\item[Message Verification]To verify a signed message $(b, k^0_b, k^1_b,...,k^{M-1}_b)$, a recipient can use $f$ and SWAP tests \cite{PhysRevLett.87.167902} to determine if $|f_{k^i_b}\rangle$ matches the corresponding public key. The number of times a recipient finds that the SWAP test fails is noted as $s$. If $s \leq c_1M$ then the recipient accepts the message (\textbf{1-ACC}), and if $s \geq c_2M$ then the recipient rejects the message (\textbf{REJ}). In the case where $c_1 < s \leq c_2$, the recipient notes that the message is valid but might not be transferable (\textbf{0-ACC}).
\end{description}

Security against forgery is possible due to Holevo's theorem \cite{Hol73}, which states that even if an eavesdropper managed to acquire all $T$ copies of the public key, they could only extract at most $Tn$ bits of data, and thus their probability of correctly guessing a single $L$-bit string of the signing key is $2^{-(L-Tn)}$. If the distance between two public keys $|\langle f_k|f_{k'}\rangle| \leq \delta$ for $k \neq k'$, then the probability of an attacker managing to make a recipient accept a signature they weren't able to correctly guess is at most $\delta^2$. Thus by choosing $c_2$ smaller than the probability of a guessing adversary being accepted $(1 - \delta^2)(1 - 2^{1 - (L-Tn)})$, we can ensure that each recipient rejects a message with high probability.

While protection against forgery is simple enough, it is harder to protect against repudiation. This is because in a two-recipient case, Alice can send Bob and Charlie different states. Due to the no-cloning theorem \cite{WZ82}, there is no perfect way for Bob and Charlie to check that their keys are the same. Gottesman and Chuang's solution is to introduce another step to the protocol, called Key Distribution, which uses a distributed SWAP test to check that the keys sent to Bob and Charlie match:

\begin{description}
\item[Key Distribution]Alice sends two copies of each key to Bob and Charlie. Bob and Charlie pick random indices $i \in \{0,...,M-1\}$ to perform a distributed SWAP on. Some of these indices they will perform SWAP tests on their own copies of the key to ensure that those match, and others Charlie will send a copy of their key to Bob so that Bob can perform a SWAP test to ensure that Bob's keys match Charlie's. If any tests fail then the protocol is aborted, otherwise they discard the test keys and continue the protocol with the remaining keys.
\end{description}

With this step in place, the probability of Alice being able to successfully repudiate is the probability that all distributed SWAP tests pass yet $|s_B - s_C| > (c_2 - c_1)M$, where $s_B$ and $s_C$ are Bob and Charlie's incorrect counts, respectively. This probability can be exponentially small, depending on how large the gap between $c_1$ and $c_2$ is.

While Lamport's signature scheme assumes the existence of classical one-way functions, no such assumption is required here. This is because quantum one-way functions can be devised, by mapping the $L$-bit strings to $n$-qubit states such that the states are nearly orthogonal, meaning that $|\langle f_k|f_k \rangle| \leq \delta$. It was shown by Buhrman et al.~\cite{PhysRevLett.87.167902} that if $\delta \approx 0.9$ then $L = 2^n$. Thus using fingerprint states as our one-way function makes it possible to create an exponentially small probability of forgery.

However, this protocol has a number of disadvantages as well:

\begin{enumerate}
\item Verification keys cannot simply be copied due to the no-cloning theorem \cite{WZ82}, so all of the keys must come from Alice. This requires authenticated quantum channels from Alice to each recipient, to ensure a forger doesn't simply send their own keys. But for generic quantum states, this is impossible \cite{1181969}, thus offering us an even harder challenge than the classical authentication we're trying to solve.
\item The recipients need to store the keys indefinitely until Alice uses them to sign a message. This requires long-time quantum memory, which is currently impractical.
\item Generating and sharing a quantum one-way function, while possible, is a complex task.
\item The distributed SWAP tests are not efficient.
\end{enumerate}

Over the rest of this section, we will summarise how these problems have been overcome in the years following this result.

\subsection{Initial Improvements}

Improvements were initially made to digital signatures by Dunjko, Wallden and Andersson \cite{PhysRevLett.112.040502}, who proposed a system which removed the need for quantum memory, the quantum SWAP test and the cost of sending quantum one-way functions by encoding the keys in coherent states. Their protocol for three parties was devised as follows, with the constraint that $0 \leq s_b < s_c < 1$:

\begin{description}
\item[Key Generation]Alice generates two $L$-bit keys $k^0, k^1$ at random for some security parameter $L$, which act as Alice's signing keys. Alice then generates states of the form $|(-1)^{k^b_l}\alpha\rangle$ for $l \in \{0,...,L-1\}$ and $b \in \{0, 1\}$, where $\alpha$ is a positive real number known between all parties and $k^b_l$ is the $l$-th bit of $k^b$. These quantum states will act as the verification key.
\item[Key Distribution]For two recipients, Bob and Charlie, Alice generates a copy of the above state for each participant. Bob and Charlie run their states together through a multiport. For states which come out of the signal port, they then use Unambiguous State Discrimination (USD) \cite{Ivanovic1987257} to distinguish between the states $\{|-\alpha\rangle, |\alpha\rangle\}$, and for each unambiguous state $|(-1)^{k^b_l}\alpha\rangle$ noting the value of $k^b_l$, the index $l$ and whether $b = 0$ or $1$.
\item[Message Signing]To sign a message $b$, Alice sends the message $(b, k^b)$ to Bob.
\item[Message Verification]To verify a message-signature pair $(b, k^b)$, Bob (resp.\ Charlie) checks the bits of $k^b$ sent to him with the bits that he was able to unambiguously determine from the distribution stage. If the number of mismatches is at most $s_bp_{USD}L$ (resp.\ $s_cp_{USD}L$), where $s_b$ (resp.\ $s_c$) are acceptance thresholds and $p_{USD}$ is the USD success probability, then Bob (resp.\ Charlie) accepts the pair.
\end{description}

The use of the multiport is to defend against repudiation; even if Alice sends $|\alpha\rangle$ to Bob and $|\beta\rangle$ to Charlie such that $\alpha \neq \beta$, only using photons that come out the signal outputs of the multiport ensures that both recipients finish with the state $|(\alpha + \beta)/2\rangle$. Because they have the same reduced state, on average they will see the same number of mismatches. Alice's best strategy has probability $(s_b + s_c)p_{USD}/2$ of making Bob and Charlie disagree on a single bit of the key, which leads to her probability of successful repudiation being negligible in terms of the difference between Charlie and Bob's thresholds, $p_{USD}$ and $L$.

In order for Bob -- or alternatively Charlie -- to successfully forge a message, Bob must guess the bits that he was not able to unambiguously get from Alice, yet Charlie was able to. The probability of this is negligible in terms of $s_c, p_{USD}$ and $L$. Bob's other option is to send wrong states through the multiport, which Charlie can also defend against by aborting if too many photons are seen through his null output\footnote{Note that if all parties are honest, the state at both null outputs is the vacuum state, so few if any photons should be seen.}.

\subsection{Quantum Digital Signatures using QKD}
\label{ssec:qds-using-qkd}

Wallden et al.\ \cite{PhysRevA.91.042304} developed in 2015 an improved signature scheme that no longer required the use of a multiport for Bob and Charlie. This scheme utilises quantum state elimination \cite{PhysRevA.89.022336} to build trust in Alice's signature. The protocol is explained below, using $|0\rangle, |1\rangle, |+\rangle = (|0\rangle + |1\rangle)/\sqrt{2}, |-\rangle = (|0\rangle - |1\rangle)/\sqrt{2}$ are the BB84 \cite{BB84} states, and $0 \leq s_b < s_c < 1$:

\begin{description}
\item[Key Generation]For a future one-bit message, Alice generates pairs $(k^0_l, k^1_l), 0 \leq l < L$ where $L$ is a security parameter and $k \in \{0, 1, +, -\}$ is chosen at random. She then generates two copies of the states $\bigotimes_{l=0}^{L-1}|k^0_l\rangle$ and $\bigotimes_{l=0}^{L-1}|k^1_l\rangle$ to act as her verification keys.
\item[Key Distribution]Alice sends copies of the verification keys to Bob and Charlie. Bob (resp.\ Charlie) picks between $L/2-r$ and $L/2+r$ indices $0 \leq l < L$ and bits $b \in \{0, 1\}$ for some $r$ and sends $(l, b, |k^b_l\rangle)$ to Charlie (resp.\ Bob), aborting if he receives a number of tuples outside of that limit. Bob and Charlie then measure all the qubits they have in either the $\{|0\rangle, |1\rangle\}$ or $\{|+\rangle\, |-\rangle\}$ basis. Bob and Charlie use the outcome of these measurements to derive what values they know the key \textit{cannot} be.
\item[Message Signing]For a message $b$, Alice sends the signed message $(b, k^b_0k^b_1...k^b_{L-1})$.
\item[Message Verification]To verify the signed message $(b, k^b_0k^b_1...k^b_{L-1})$, Bob (resp.\ Charlie) checks how many of the symbols in the signing key are in the set of values they eliminated from measurement and accepts if this number is smaller than $s_bL$ (resp.\ $s_cL$).
\end{description}

Because Alice does not know which qubits Bob measured and which he forwarded to Charlie (and vice versa), she gains no repudiation advantage from sending different states to the two receivers. Her best strategy is to send a signed message which has $L(s_c - s_b)L/2$ mismatches with the actual signature, which has a negligible chance of success in terms of $L$. There is also a negligible chance of forgery, as even if Bob faked the qubits he was forwarding on to Charlie, he would still need to guess the qubits that Charlie didn't forward on to him.

\subsection{Removing the need for Authenticated Quantum Channels}

2016 was when we finally saw quantum digital signatures proposed without authenticated quantum channels, in two papers: one by Yin, Fu and Chen \cite{PhysRevA.93.032316} and one published a week later by Amiri et al.\ \cite{PhysRevA.93.032325}. The latter paper will be the focus of this section.

This signature scheme relies on a key generation protocol; a protocol undertaken between Alice and each receiver independently. Based upon BB84 \cite{BB84}, the key generation protocol used between Alice and, for example, Bob is described below:

\begin{enumerate}
\item Bob generates qubits in either $|0\rangle, |1\rangle, |+\rangle \text{ or } |-\rangle$ states and forwards these states on to Alice.
\item Alice measures each state in either the $X$ or $Z$ basis, noting her chosen basis and results.
\item Alice and Bob publicly announce their choice of bases and throw out any results that were measured in different bases. Steps 1-3 are repeated until enough bits are produced.
\item Bob divides his key up into four parts of equal size:
\begin{itemize}
\item The first set of bits, $V$, are all results from $X$ basis measurements, and will be compared with the corresponding bits from Alice's measurements to see how much correlation there is between their bits.
\item The second set of bits, $Z$, are all results from $Z$ basis measurements, and will be compared with Alice's corresponding bits to determine how much of their communications has been eavesdropped upon.
\item The third set of bits, $C^B$ are bits that are forwarded on to Charlie.
\item The fourth set of bits, $B$ are the bits that Bob keeps for his own private key. All other bits are discarded by both him and Alice.
\end{itemize}
\item If there is not enough correlation or the difference in the amount of correlation between the $V$ and $Z$ bits is too high, then the protocol is aborted.
\item Otherwise, Bob and Charlie exchange their parts of the keys through a secure \& authenticated classical channel.
\end{enumerate}

This protocol allows Alice to generate highly correlated keys with the recipients of her message. We use this to develop a digital signature scheme as follows, with $0 \leq s_b < s_c < 1/2$ and $L$ as our security parameter:

\begin{description}
\item[Key Generation/Distribution]Bob and Charlie use the key generation protocol to generate two $L$-bit keys with Alice, half of each key being exchanged with the other party. Alice's copies of the keys are denoted $A^B_0, A^B_1, A^C_0 \text{ and } A^C_1$, where $B$ (resp.\ $C$) denotes Bob (resp.\ Charlie). Bob and Charlie have their own keys with Alice $B_0, B_1$ \& $C_0, C_1$ and their keys from each other $B^C_0, B^C_1$ \& $C^B_0, C^B_1$, respectively.
\item[Message Signing]To sign a bit $b$, Alice sends the message $(b, A^B_b, A^C_b)$.
\item[Message Verification]To verify a sent message message $(b, A^B_b, A^C_b)$, Bob checks the number of bits in $A^B_b$ that mismatch $B_b$ and the number of bits in $A^C_b$ that mismatch $B^C_b$. If the total number of mismatches is below $s_bL/2$ then the message is rejected. Otherwise the message is accepted.
\end{description}

This signature is a more general version of a protocol from 2015. While the protocol described above uses a key generation protocol that creates correlated bit strings between parties, protocol P2 in \cite{PhysRevA.91.042304} uses already shared keys which perfectly match.

Like the protocol in SEC.\ \ref{ssec:qds-using-qkd}, security against repudiation is offered as Alice does not know which bits of their respective keys Bob and Charlie passed on to each other.

For Bob to forge a message to send to Charlie, he needs to guess the values of the bits that Charlie did not forward on to him, which is negligible for the same reason as in SEC.\ \ref{ssec:qds-using-qkd}. Bob has the highest chance of success if he eavesdrops on the key generation protocol between Alice and Charlie, which he now can given that the channel is not authenticated. If Bob regularly guesses the wrong qubit to forward to Alice, this will result in low correlations between Alice and Charlie's measurements in the $X$ or $Z$ basis. Thus when Alice and Charlie look at the correlations between their $V$ and $X$ parts of their keys, they will either find low correlations in $V$ or a significant difference between the number of correlations in $V$ and the number of correlations in $X$. Either way, the protocol will be aborted. Thus the probability of a successful forgery is negligible.

\section{Experimental Achievements}

\section{Conclusion and Discussion}

\bibliography{qde}% Produces the bibliography via BibTeX.

\appendix

\section{Arbitrated Quantum Signatures}

\end{document}
%
% ****** End of file qde.tex ******

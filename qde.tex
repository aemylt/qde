% TeX'ing this file requires that you have AMS-LaTeX 2.0 installed
% as well as the rest of the prerequisites for REVTeX 4.1
%
% See the REVTeX 4 README file
% It also requires running BibTeX. The commands are as follows:
%
%  1)  latex qde.tex
%  2)  bibtex qde
%  3)  latex qde.tex
%  4)  latex qde.tex
%
\documentclass[%
 reprint,
%superscriptaddress,
%groupedaddress,
%unsortedaddress,
%runinaddress,
%frontmatterverbose, 
%preprint,
%showpacs,preprintnumbers,
%nofootinbib,
%nobibnotes,
%bibnotes,
 amsmath,amssymb,
 aps,
 pra,
%prb,
%rmp,
%prstab,
%prstper,
%floatfix,
]{revtex4-1}

\usepackage{graphicx}% Include figure files
\usepackage{dcolumn}% Align table columns on decimal point
\usepackage{bm}% bold math
\usepackage{hyperref}% add hypertext capabilities
%\usepackage[mathlines]{lineno}% Enable numbering of text and display math
%\linenumbers\relax % Commence numbering lines

%\usepackage[showframe,%Uncomment any one of the following lines to test 
%%scale=0.7, marginratio={1:1, 2:3}, ignoreall,% default settings
%%text={7in,10in},centering,
%%margin=1.5in,
%%total={6.5in,8.75in}, top=1.2in, left=0.9in, includefoot,
%%height=10in,a5paper,hmargin={3cm,0.8in},
%]{geometry}

\usepackage{amsthm}
\newtheorem{definition}{Definition}[section]

\begin{document}

\title{Quantum Digital Signatures}

\author{Dominic Moylett}
 \email{dominic.moylett@bristol.ac.uk}
\affiliation{%
 Quantum Engineering Centre for Doctoral Training\\
 University of Bristol
}%

\date{\today}% It is always \today, today,
             %  but any date may be explicitly specified

\begin{abstract}
While classical cryptography has relied on the assumed existence of one-way functions for proofs of security, quantum cryptography has given us information theoretically secure secret sharing. This has been achieved via quantum key distribution, utilising the uncertainty principle to limit what an eavesdropper can discover about a shared private key. Another application where we can have provably secure primitives is that of quantum digital signatures, which are used for authenticating that a message did come from a specific user. In this essay, we summarise the theoretical and experimental achievements in quantum digital signatures, and discuss their current limitations.
\end{abstract}

%\pacs{Valid PACS appear here}% PACS, the Physics and Astronomy
                             % Classification Scheme.
%\keywords{Suggested keywords}%Use showkeys class option if keyword
                              %display desired
\maketitle

%\tableofcontents

\section{Introduction}

\section{Preliminaries}

\begin{definition}
A function $f:\mathbb{N}\rightarrow\mathbb{R}$ is negligible iff $\forall~ c > 0 ~\exists~ n_0 \in \mathbb{N}$ such that $\forall~ n \geq n_0, |f(n)| < \frac{1}{n^c}$.
\end{definition}

The most common example of negligible functions are functions of the form $2^{-p(n)}$ for some positive polynomial $p$.

\section{Definition of Digital Signatures}

A digital signature scheme consists of the following stages:

\begin{description}
\item[Key Generation]Alice generates a signing key $sk$ and a verification key $vk$. The signing key is kept to herself, while the verification key is shared with other parties.
\item[Message Signing]Alice takes a message $m$ and her signing key $sk$ and produces a signature for that message $s_m$.
\item[Message Verification]Another party, Bob, takes a message $m$, Alice's verification key $vk$ and a signature $s_m$ and accepts if they think the message was sent and signed by Alice, otherwise they reject.
\end{description}

Another definition of verification used in, for example, \cite{quant-ph/0105032} has three different options for Bob. In this definition Bob can output:

\begin{description}
\item[1-ACC]He thinks the message-signature pair is valid and other parties will also find the message to be valid.
\item[0-ACC]He thinks the message-signature pair is valid and other parties will not find the message to be valid.
\item[REJ]He thinks the message-signature pair is invalid.
\end{description}

For security, we care about two properties with digital signatures:

\begin{description}
\item[Repudiation]The probability that Alice is able to convince Bob that a message-signature pair is valid and convince another party Charlie that the same pair is invalid is negligible in terms of the security parameter (often the length of the signature).
\item[Forgery]The probability that Bob can in polynomial time create a valid message-signature pair he has not previously seen sent by Alice is negligible in terms of the security parameter.
\end{description}

\section{Lamport's One-Time Digital Signature}

The original quantum digital signature was inspired from a classical family of digital signatures. This family of signatures were first published by Lamport\cite{lamp79}, and are reliant on the existence of one-way functions, defined below.

\begin{definition}
Let $\mathcal{X}, \mathcal{Y}$ be arbitrary sets. A function $f:\mathcal{X} \rightarrow \mathcal{Y}$ is one-way iff $f$ can be computed in polynomial time, but for any polynomial-time randomised adversary $f^{-1}:\mathcal{Y} \rightarrow \mathcal{X}$ and uniformly selected $x \in \mathcal{X}, \mathrm{Pr}[f(f^{-1}(x)) = f(x)]$ is negligible in terms of the security parameter (often the length of $x$ in bits).
\end{definition}

For a one-way function $f$, then Lamport's one-time digital signature scheme is defined as follows for Alice signing an $m$-bit message:

\begin{description}
\item[Key Generation]Alice generates $m$ pairs of uniformly selected random integers $\{k^i_0, k^i_1\}$. These will act as Alice's signing key, while the verification key will be the pairs $\{f(k^i_0), f(k^i_1)\}$ for $i \in \{0,...,m-1\}$.
\item[Message Signing]For each bit $m_i$ of Alice's message, she sends $k^i_{m_i}$ as her signature for that bit.
\item[Message Verification]Given Alice's verification key $\{v^i_0, v^i_1\}$ and her signature $k^i_{m_i}$, her message $m_i$ can be verified by accepting if $v^i_{m_i} = f(k^i_{m_i})$ and rejecting otherwise.
\end{description}

Because this signature scheme is classical and deterministic, it is trivial for any verifying parties to share their copies of the message and verification keys and thus impossible for Alice to repudiate the message. As for forgery, if it is possible for an adversary to forge a message given only one message from Alice then it is possible to invert $f$, thus $f$ is not a one-way function.

Note that this signature scheme can only be used for verifying a single $m$-bit message. This is because if an attacker has signatures for two different messages $M, M'$ then they can construct the valid signature for the new message $m_0m_1...m'_i...m_{m-1}$, where $m_i \neq m'_i$.

\section{Quantum Digital Signatures}

\subsection{Gottesman and Chuang}

The first quantum digital signature was devised in 2001 by Daniel Gottesman and Isaac Chuang \citep{quant-ph/0105032}. The protocol used the same principles as Lamport's one-time signature scheme, but utilised quantum one-way functions, which map classical $L$-bit strings $k$ to $n$-qubit quantum states $|f_k\rangle$. The full protocol is described below, where $f$ is the quantum one-way function, $c_1, c_2$ are security thresholds such that $0 \leq c_1 < c_2 < 1$, $M$ is the security parameter, and $T < L/n$ is the number of copies of each key:

\begin{description}
\item[Key Generation]To sign a single bit, Alice generates pairs of $L$-bit strings $\{k^i_0, k^i_1\}, i \in \{0,...,M-1\}$ uniformly at random to act as her private key. Her public key is the pairs of quantum states $\{|f_{k^i_0}\rangle, |f_{k^i_1}\rangle\}, i \in \{0,...,M-1\}$.
\item[Message Signing]To sign a bit $b$, Alice sends the tuple $(b, k^0_b, k^1_b,...,k^{M-1}_b)$.
\item[Message Verification]To verify a signed message $(b, k^0_b, k^1_b,...,k^{M-1}_b)$, a recipient can use $f$ and SWAP tests \cite{PhysRevLett.87.167902} to determine if $|f_{k^i_b}\rangle$ matches the corresponding public key. The number of times a recipient finds that the SWAP test fails is noted as $s$. If $s \leq c_1M$ then the recipient accepts the message (\textbf{1-ACC}), and if $s \geq c_2M$ then the recipient rejects the message (\textbf{REJ}). In the case where $c_1 < s \leq c_2$, the recipient notes that the message is valid but might not be transferable (\textbf{0-ACC}).
\end{description}

Security against forgery is possible due to Holevo's theorem \cite{Hol73}, which states that even if an eavesdropper managed to acquire all $T$ copies of the public key, they could only extract at most $Tn$ bits of data, and thus their probability of correctly guessing a single $L$-bit string of the signing key is $2^{-(L-Tn)}$. If the distance between two public keys $|\langle f_k|f_{k'}\rangle| \leq \delta$ for $k \neq k'$, then the probability of an attacker managing to make a recipient accept a signature they weren't able to correctly guess is at most $\delta^2$. Thus by choosing $c_2$ smaller than the probability of a guessing adversary being accepted $(1 - \delta^2)(1 - 2^{1 - (L-Tn)})$, we can ensure that each recipient rejects a message with high probability.

While protection against forgery is simple enough, it is harder to protect against repudiation. This is because in a two-recipient case, Alice can send Bob and Charlie different states. Due to the no-cloning theorem \cite{WZ82}, there is no perfect way for Bob and Charlie to check that their keys are the same. Gottesman and Chuang's solution is to introduce another step to the protocol, called Key Distribution, which uses a distributed SWAP test to check that the keys sent to Bob and Charlie match:

\begin{description}
\item[Key Distribution]Alice sends two copies of each key to Bob and Charlie. Bob and Charlie pick random indices $i \in \{0,...,M-1\}$ to perform a distributed SWAP on. Some of these indices they will perform SWAP tests on their own copies of the key to ensure that those match, and others Charlie will send a copy of their key to Bob so that Bob can perform a SWAP test to ensure that Bob's keys match Charlie's. If any tests fail then the protocol is aborted, otherwise they discard the test keys and continue the protocol with the remaining keys.
\end{description}

With this step in place, the probability of Alice being able to successfully repudiate is the probability that all distributed SWAP tests pass yet $|s_B - s_C| > (c_2 - c_1)M$, where $s_B$ and $s_C$ are Bob and Charlie's incorrect counts, respectively. This probability can be exponentially small, depending on how large the gap between $c_1$ and $c_2$ is.

While Lamport's signature scheme assumes the existence of classical one-way functions, no such assumption is required here. This is because quantum one-way functions can be devised, by mapping the $L$-bit strings to $n$-qubit states such that the states are nearly orthogonal, meaning that $|\langle f_k|f_k \rangle| \leq \delta$. It was shown by Buhrman et al.~\cite{PhysRevLett.87.167902} that if $\delta \approx 0.9$ then $L = 2^n$. Thus using fingerprint states as our one-way function makes it possible to create an exponentially small probability of forgery.

However, this protocol has a number of disadvantages as well:

\begin{enumerate}
\item Verification keys cannot simply be copied due to the no-cloning theorem \cite{WZ82}, so all of the keys must come from Alice. This requires authenticated quantum channels from Alice to each recipient, to ensure a forger doesn't simply send their own keys. But for generic quantum states, this is impossible \cite{1181969}, thus offering us an even harder challenge than the classical authentication we're trying to solve.
\item The recipients need to store the keys indefinitely until Alice uses them to sign a message. This requires long-time quantum memory, which is currently impractical.
\item Generating and sharing a quantum one-way function, while possible, is a complex task.
\item The distributed SWAP tests are not efficient.
\end{enumerate}

Over the rest of this section, we will summarise how these problems have been overcome in the years following this result.

\subsection{Dunjko, Wallden and Andersson}

\subsection{Wallden, Dunjko, Kent and Andersson}

\subsection{Yin, Fu and Chen}

\subsection{Amiri, Wallden, Kent and Andersson}

\section{Experimental Achievements}

\section{Current Limitations}

\section{Conclusion}

\bibliography{qde}% Produces the bibliography via BibTeX.

\end{document}
%
% ****** End of file qde.tex ******
